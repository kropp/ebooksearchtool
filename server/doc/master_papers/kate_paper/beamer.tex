\documentclass[handout]{beamer}
\usepackage{presentation}
%\setbeamercovered{dynamic}
\title{Набор инструментов для поиска электронных книг в Интернете. }
\subtitle{Серверная часть, обеспечивающая верификацию, обновление данных и взаимодействие с клиентскими программами и пользователем.}
\author{Екатерина Тузова}
\institute{АУ РАН}
\date{\today}


\begin{document}

\begin{frame}
%  \transdissolve[duration=0.2]
  \titlepage
\end{frame}

%\section*{План презентации}

\begin{frame}
\section*{План презентации}
%  \transdissolve[duration=0.2]
  \frametitle{О чем пойдет речь...}
  \tableofcontents[pausesections]
\end{frame}
%\section{Визуальное оформление}
%\subsection{Блоки}

\begin{frame}
%  \transdissolve[duration=0.2]
\section{Визуальное оформление}
  \frametitle{Это заголовок фрейма}
  \begin{block}{Заголовок блока}
  Внутри блока мы можем писать любой текст и даже формулы.
  Например: интегральная формула Коши для точки $z_0$
  внутри замкнутого контура $\ell$ имеет вид
  \end{block}
  \section{3}
  фйуцк
    \section{4}
  %
  \begin{block}{}
  Блок без заголовка тоже имеет право на жизнь
  \end{block}
  %
  А этот текст вне блока, просто во фрейме.
\end{frame}
\end{document}