\cleardoublepage
\section{Библиографический список}
 
\renewcommand*{\refname}{}
\begin{thebibliography}{99}

% For head
\bibitem{googleBook} Сервис для полнотекстового поиска по книгам Google Books \\ \url{http://books.google.com/}

\bibitem{ebdb} База данных электронных книг eBdb\\  \url{http://ebdb.ru/}

\bibitem{kindle} Программно-аппаратная платформа для чтения электронных книг Amazon Kindle\\ \url{http://amazon.com/kindle/}

\bibitem{sonyreader} Программно-аппаратная платформа для чтения электронных книг Sony Reader\\ \url{http://www.learningcenter.sony.us/assets/itpd/reader/}

\bibitem{opds} Описание стандарта The Open Publication Distribution System (OPDS)\\ \url{http://code.google.com/p/openpub/wiki/CatalogSpecDraft}

\bibitem{bookserver} Открытая система распостранения электронных книг BookServer\\ \url{http://bookserver.archive.org/}

\bibitem{archive} Библиотека Internet Archive\\ \url{http://archive.org/}

\bibitem{feedbooks} Сервер предоставляющий доступ к библиотеке в формате OPDS \\ \url{http://feedbooks.com/}

\bibitem{django} Страница проекта веб-фреймворк Django\\ \url{http://www.djangoproject.com/}

\bibitem{djangomvc} Описание архитектуры Django сайтов на DjangoBook\\ \url{http://www.djangobook.com/en/2.0/chapter01/}

% for kate
\bibitem{machine-learning} Страница про Машинное обучение на английской Wikipedia\\ \url{http://en.wikipedia.org/wiki/Machine_learning}

\bibitem{collective-intelligence} Тоби Сегаран. Программируем коллективный разум. Спб: Символ-Плюс, 2008. 312с.

\bibitem{stemming} Страница про стемминг на английской Wikipedia\\ \url{http://en.wikipedia.org/wiki/Stemming/}

\bibitem{amazon} Amazon. Интернет-сервис, ориентированный на продажу реальных товаров массового спроса \\ \url{http://amazon.com/}

\bibitem{wiki} Wikipedia. Свободная общедоступная многоязычная универсальная интернет-энциклопедия \\ \url{http://wikipedia.org/}

\bibitem{allrom} Сервер предоставляющий доступ к библиотеке в формате OPDS \\
\url{http://www.allromanceebooks.com/}

\bibitem{smash} Сервер предоставляющий доступ к библиотеке в формате OPDS \\
\url{http://www.smashwords.com/}

\bibitem{beaut-soup} Библиотека Beautiful Soup \\ \url{http://www.crummy.com/software/BeautifulSoup/documentation.html}

\bibitem{epub} Страница про формат ePub на английской Wikipedia\\ \url{http://en.wikipedia.org/wiki/Epub/}

\bibitem{fb2} Страница про формат fb2 на английской Wikipedia\\ \url{http://en.wikipedia.org/wiki/Fb2/}

\bibitem{doc-admin} Документация проекта Django для интерфейса администратора\\ \url{http://docs.djangoproject.com/en/dev/ref/contrib/admin/}

\bibitem{djbook-admin} Описание интерфейса администратора в DjangoBook\\ \url{http://www.djangobook.com/en/1.0/chapter06/}

\bibitem{many-to-many} Страница про поля типа многие-ко-многим на английской Wikipedia\\ \url{http://en.wikipedia.org/wiki/Many-to-many}

\bibitem{mvc} Страница про паттерн Модель - Представление - Управление на английской Wikipedia\\
\url{http://en.wikipedia.org/wiki/Model-view-controller}

\bibitem{mvc-djangobook} Описание концепции Модель - Представление - Управление в DjangoBook \\
\url{http://www.djangobook.com/en/2.0/chapter01/}

\bibitem{firebug} Плагин firebug для firefox \\
\url{http://getfirebug.com/}

\bibitem{cron} Статья о демоне-планировщике cron на английской Wikipedia \\
\url{http://en.wikipedia.org/wiki/Cron}


\end{thebibliography}
