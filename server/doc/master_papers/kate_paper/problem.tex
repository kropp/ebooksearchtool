\section{Постановка задачи}

Основой для данной работы послужили задачи поиска дополнительной информации, а так же верификация данных, их структурирование  и предоставление пользователю и клиентским программам.


\subsection{Верификация и обновление данных}

После того, как книги найдены и добавлены в базу данных требуется все время поддерживать ее в рабочем состоянии, обновлять имеющуюся инфомацию и производить поиск дополнительной.

Верификация нужна потому, что информация, поступающая от <<поискового паука>> (crawler) и анализатора (analyzer) зачастую бывает весьма сомнительного качества. Это происходит из-за того, что иногда <<поисковый паук>> находит книги, находящиеся на <<сложных>> страницах. С таких страниц анализатор либо не может извлечь информацию об авторе и названии книги, либо извлекает ее с ошибками. 

Если анализатор не может определить автора/название, то в базу добавляется только ссылка на файл. Позже необходимо извлечь максимум доступной информации либо из самого файла, если это возможно, либо из других источников.

Часто возникают ситуации, когда со страницы с найденной книгой возможно извлечь только информацию об авторе и названии книги. Сервер же ориентирован на предоставление информации о книгах в удобном для пользователя виде. Одним из таких способов является разбиение книг по таким каталогам, как различные языки и жанры. Для обеспечения лучшего разбиения по таким каталогам необходимо реализовать поиск дополнительной информации, такой как аннотация, теги и пр., для книг и авторов, которые уже находятся в базе.


Так же существует вероятность, что информация об авторе и названии книги извлечена со станицы с ошибками. В таком случае необходимо предоставить возможность администратору исправлять её достаточно простым способом, например, через веб-интерфейс.


\subsection{Предоставление данных}

Сервер должен предоставлять информацию двумя способами --- в HTML виде для пользователей, желающих использовать веб-browser, и по протоколу OPDS --- для клиентских программ. Предоставление данных должно быть удобным для пользователя. Должен присутствовать простой и расширенный поиск, а так же разделение данных по каталогам. Для предоставления информации необходимо описать логику отображения информации и способы ее отображения. Необходимо реализовать поддержку протокола OPDS, а также поддержку технологии поиска OpenSearch.

\newpage