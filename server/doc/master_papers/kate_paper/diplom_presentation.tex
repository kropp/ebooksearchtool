\documentclass[handout]{beamer}
\usepackage{presentation}

\title{Сервис поиска электронных книг }
\subtitle{Серверная часть, обеспечивающая верификацию, обновление данных и взаимодействие с клиентскими программами и пользователем.}
%\author{Е. А. Тузова}
%\institute{АУ РАН}
\date{}


\begin{document}

\begin{frame}
  \titlepage

  \begin{flushright}
    Студент~~~~~~  Е. А. Тузова
  
    Руководитель~~~  Н. М. Пульцин

  \end{flushright}
\end{frame}

\section*{План}
  \begin{frame}
    \frametitle{План}
    \tableofcontents[pausesections]

  \end{frame}

\section{Постановка задачи}
  \begin{frame}

    \frametitle{Постановка задачи}
    \begin{enumerate}
      \item Верификация и обновление данных
      \item Предоставление данных пользователю и клиентским программам
    \end{enumerate}
  \end{frame}

\section{Задача обновления информации}
  \begin{frame}
    \frametitle{Задача обновления информации}
  
    \begin{enumerate}
      \item Извлечение информации из форматов epub и fb2 
      \item Определение жанра книги
      \item Поиск информации на внешних ресурсах
    \end{enumerate}

  \end{frame}
  
\section{Извлечение информации из форматов epub и fb2 }
  \begin{frame}
    \frametitle{Извлечение информации из форматов epub и fb2 }
  
  \end{frame}

\section{Определение жанра книги}
  \begin{frame}
    \frametitle{Определение жанра книги}  
    
    \begin{enumerate}
      \item Сравнение различных видов классификации
      \item Байесовский классификатор
    \end{enumerate}        
  \end{frame}

\subsection{Сравнение различных видов классификации}
  \begin{frame}
    \frametitle{Сравнение различных видов классификации}  
    
    \begin{enumerate}
      \item Деревья принятия решений
      \item Нейронные сети
      \item Метод k-средних
      \item Байесовский классификатор
    \end{enumerate}        
  \end{frame}


\section{Заключение}
  \begin{frame}
    \frametitle{Заключение}
    \begin{enumerate}
      \item Для обновления информации реализован самообучающийся классификатор, позволяющий предсказать жанр книги по её аннотации.
	  \item Реализован поиск дополнительной информации и извлечение информации из форматов ePub и FB2
	  \item Усовершенствован стандартный интерфейс администратора, предоставляемый фреймворком Django.
	  \item Реализовано предоставление данных в xHTML виде для пользователей и по протоколу OPDS для клиентских пр
    \end{enumerate}
  \end{frame}

\end{document}