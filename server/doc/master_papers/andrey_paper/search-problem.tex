\section{Интерфейс поискового механизма}

\subsection{Методы}


Для обеспечения модульности, гибкости кода реализация поиска и его использование разделено.

Отдельно описан интерфейс поискового механизма. Код, использующий поисковые функции, опирается только на этот интерфейс. За ним скрыта вся логика поиска.

При разработке конкретного поискового механизма необходимо реализовать этот интерфейс. Если в будущем понадобилось изменить или даже заменить поисковый механизм, то эти изменения не затронут остальной код программы.

Интерфейс описан классом SearchEngine.
Этот интерфейс содержит всего три метода:
\begin{verbatim}
author_search(**kwargs)
book_search(**kwargs)
simple_search(query, **kwargs)
\end{verbatim}


author\_search() осуществляет поиск по авторам, принимает имена авторов как обязательный аргумент и, возможно, дополнительные аргументы (зависит от реализации).

book\_search() осуществляет поиск по книгам, принимает название книги как обязательный аргумент, имена авторов как необязательный аргумент и, возможно, дополнительные аргументы. 

simple\_search(query) осуществляет поиск по книгам, принимает аргумент {\em запрос}, не специфицирующий сущность, к которой относится.
Этот метод обеспечивает {\em простой поиск}.
Позволяет пользователю не указывать сущность запроса (возможен смешанный запрос, например <<Пушкин Евгений Онегин>>). Допускаются дополнительные аргументы. 

Упомянутые выше {\em дополнительные аргументы} --- это аргументы типа тэг, язык книги, фильтрующие результаты поиска. 


\subsection{Возвращаемое значение}

Каждый метод возвращает итерируемую коллекцию, содержащую соответствующие сущности (авторов или книги). 

У каждого возвращаемого значения есть атрибут suggestion, содержащий словарь возможных исправлений запроса или None, если таковых не имеется. Пример: 
\begin{verbatim}
suggestion = {
    author: 'Tolstoy',
    title: 'Anna Kerenina',
}
\end{verbatim}
