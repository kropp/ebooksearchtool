%\subtitle{Задача поиска}

Если существует некоторая большая база с информацие, то очевидно, 
что для быстрой и удобной работой с ней, необходим мощный и гибкий поиск.

В нашей модели для пользователя есть несколько сущностей: название книги, 
список её авторов, язык на котором написана книга, тэги, характеризующие её стиль, жанр, содержание.

Ниже сформулированы требования для функции поиска, с точки зрения пользователя:
\begin{enumerate}
  \item  Релевантный поиск как по отдельным сущностям, так и по различным их комбинациям;
  \item  Фильтрация результатов поиска по некоторым сущностям (язык книги, тэг);
  \item  Поиск с учётом морфологии языка;
  \item  Поиск с учётом опечаток в запросе;
  \item  Простой поиск (простой в использовании).
\end{enumerate}

При разработке поискового механихма необходимо сохранить слабую связанность отдельных компонент программы. Обеспечить возможность замены реальзации поиска с минимальными изменениями в остальном коде. Учесть возможность расширение и изменения требований к функции поска.



--- Интерфейс поискового механизма ---
Для обеспечения модульнусти, гибкости кода, реализация поска и его использование разделено.

Отдельно описан интерфейс поискового механизма, код, использующий поисковые функции, опирается только на этот интерфейс. За ним скрыта вся логика поиска.

При разработке кокретного поискового механизма, необходимо реализовать этот интерефейс. Если в будущем понадобилось изменить или даже заменить поисковый механизм, то эти изменения не затронут остальной код программы.

Интерфейс описан классом SearchEngine в файле book/search\_engine/engine.py. 

Интерфейс содержит всего три метода 
\begin{verbatim}
author_search(**kwargs)
book_search(**kwargs)
simple_search(query, **kwargs)
\end{verbatim}


author\_search() осуществляет поиск по авторам, принимает имена авторов как обязательный аргумент и, возможно, дополнительные аргументы (зависит от реализации).

book\_search() осуществляет поиск по книгам, принимает название книги как аргументы обязательный, имена авторов как необязательный парамерт и, возможно, дополнительные аргументы. 

simple\_search(query) осуществляет поиск по книгам, принимает аргумент 'запрос', не специфицирующий сущность к которой относится.
Этот метод обеспечивает "простой" поиск.
Позволяет пользователю не указывать сущность запроса (возможно смешанный запрос, например "Пушкин Евгений Онегин"). Допускаются дополнительные аргументы. 

Упомянутые выше дополнительные аргументы, это аргументы типа: тэг, язык книги, фильтрующие результаты поиска. 


Возвращаемое значение

Каждый метод возвращает итерируемую коллекцию, содержащую соответствующие сущности (авторов или книги). 

У каждого возвращаемого значения есть атрибут suggestion содержащий словарь возможных исправлений запроса или None, если таковых не имеется. Пример: 
\begin{verbatim}
suggestion = {
    author: 'Tolstoy',
    title: 'Anna Kerenina',
}
\end{verbatim}
