\section{Интерфейс к анализатору}


\subsection{Алгоритм взаимодействия}

Важной частью функционирования системы в целом является процесс добавление информации на сервер со стороны анализатора. Этот процесс состоит из двух этапов. 

Первый этап --- это распознавание анализатором названия книги, её авторов и прочей информации о книге. На этом этапе анализатор обращается к серверу с запросами вида: есть ли в базе автор с именем, похожим на заданное, какие книги заданного автора есть в базе.

Второй этап --- это добавление информации на сервер. В этот момент анализатор пользуется интерфейсом модификации данных.


В разных файлах одной и той же книги могут быть по разному записаны имена авторов. Для поддержания базы в корректном состоянии необходимо распознавать такие неточности.

К сущностям {\em автор, книга, файл книги} было добавлено новое понятие {\em индекс доверия (credit)}. Это понятие характеризует проверенность информации о сущности.
Также введено понятие {\em индекс релевантности (relevance)}. Этот индекс характеризует похожесть двух строк.
Эти оба индекса возращаются при поиске по авторам и по книгам.
Анализатор принимает решение на основании значений этих двух индексов.

Если для автора индекс доверия и релевантности оказались больше пороговых значений, то не создаётся новой сущности.
В противном случае создаётся новый автор.
Если для книги индекс доверия и релевантности оказались больше пороговых значений, и авторы этой книги распознались как уже существующие в базе, то новой сущности не создаётся.

\subsection{Фаза распознования информации}

В фазе распознвания книги анализатору необходимо знать, какие авторы, книги уже присутствуют в базе на сервере.
В результатах поиска по названиям книг или по именам авторов необходимо к каждой сущности указывать индекс доверия и релевантности.
Индекс доверия хранится в базе данных, поэтому с ним нет больших трудностей.
А индекс релевантности необходимо расчитывать для каждого запроса.

Для этого была реализована оболочка над поисковым интерфейсом.
Сначала производится поиск по заданной сущности с использованием одной из реализаций поискового механизма (в этой работе поисковый механизм основанный на Sphinx). Затем для каждой сущности из результатах поиска вычисляется индекс релевантности.
В этой задаче необходимо сравнивать похожесть двух строк состоящих из одного или более слов.
Для этого было применено модифицированное расстояние Левенштейна.
Алгоритм Левенштейна \cite{distance} неплохо работает для названий книг одного автора.
Но весьма неудовлетворительно для имён людей.

Расстония Левенштейна между строками <<Лев Толстой>> и <<Толстой Лев Николаевич>> будет больше, 
чем расстояние между <<Лев Толстой>> и <<Александр Толстой>>.
Очевидно, что первая пара имён, скорее всего, соответсвует одному и тому же человеку. Вторая пара --- скорее различным людям.

Анализ различных написаний имён авторов показал, что часто в имени автора указываются не все части плоного имени.
Например, часто опускают отчество (или среднее имя), порой и имя.
Также встречается различный порядок написания частей полного имени автора.

Основная идея модифицированный алгоритма Левенштейна --- это изменить его поведения на строках состоящих из нескольких слов.

Алгоритм основывается на следующих предположениях:
\begin{enumerate}
	\item порядок следования слов в строке не очень важен,
	\item цена удаления/удаления слова из сторки скорее всего меньше, чем её длинна,
	\item расстояние между произвольными парами строк \textit{Q} и \textit{$S_{1}$} меньше, 
	  	  чем расстояние между строками \textit{Q} и \textit{$S_{2}$},
		  если у \textit{Q} и \textit{$S_{1}$} совпало больше слов, чем у \textit{Q} и \textit{$S_{2}$}.
\end{enumerate}


%\rule{1}
\algo{Расчёт расстояния между строками}
{
Берутся две строки \textit{$s_{1}$} и \textit{$s_{2}$} и разбиваются на слова, так \textit{$S_{1}$} --- набор слов из \textit{$s_{1}$}.

Для каждой пары слов \textit{a}, \textit{b}, \textit{a}$\subset$\textit{$S_{1}$}, \textit{b}$\subset$\textit{$S_{2}$} вычисляется  расстояние Левенштейна.
Таким образом получается {\em матрица расстояний} \textit{M} размера \textit{n}$\times$\textit{m}.

Затем вычисляется минимальная сумма $\Sigma$ из \textit{min(n, m)} элементов матрицы \textit{M} таких,
что ни один из элементов не стоит ни в одном столбце, ни на одной строчки с другими элементами.
$\Sigma$ --- это сумма цен изменения слов из набора \textit{$S_{1}$} в слова из набора \textit{$S_{2}$} без учёта порядка следования слов
и их удаления/добавления.

Прибавив к $\Sigma$ цену добавления/удаления слов \textit{abs(n-m) $\ast$ $C_{remove}$}, где \textit{$C_{remove}$} цена удаления/доваления одного слова, получаем минимальное расстояние между набором слов \textit{$S_{1}$} и \textit{$S_{2}$}.
}


\subsection{Фаза добавления книги}


Для анализатора реализован интерфейс, позволяющий добавлять и модифицировать новые сущности.

Запрос состоит из двух секций: define и update. 

\begin{verbatim}
<?xml version="1.0" encoding="UTF-8"?>
<request>
    <define>
        ...
    </define>

    <update>
        ...
    </update>
</request>
\end{verbatim}


\subsubsection{Секция define}

Здесь необходимо описать создаваемые сущности, но не уже существующие в базе. У каждой сущности должен быть атрибут --- {\em уникальный идентификатор ui} (целое положительное число). Это уникальный идентификатор сущности для этого запроса. В следующей секции по этим ui можно обращаться к сущностям. 

При определении каждой сущности необходимо указать обязательную информацию: \\
для author -- full\_name \\
для file -- link, size, type\\
для book -- title \\


Можно добавить необязательную информацию: credit, lang, ... 

Пример определения автора 
\begin{verbatim}
<author ui="1">
    <full_name> Leo Tolstoy </full_name>
</author>
\end{verbatim}

Пример определения файла книги 
\begin{verbatim}
<file ui="2">
    <link>http://example.com</link>
    <type>pdf</type>
    <size>4563214</size>
</file>
\end{verbatim}

Пример определения книги 
\begin{verbatim}
<book ui="3">
    <title> Red hat </title>
</book>
\end{verbatim}

\subsubsection{Секция update}

В этой секции возможна модификация данных. 

К каждой из трёх сущностей возможен доступ по id, если эта сущность уже существует в базе, либо по ui, если она создается в этом запросе. 

Вся указаная информация о сущности будет либо перезаписана, либо добавлена. 

Пример изменения имени автора 
\begin{verbatim}
<author id="45">
    <full_name> Alexander Pushkin </full_name>
</author>
\end{verbatim}

Пример создания новой книги 
\begin{verbatim}
<book ui="3">
    <authors>
        <author id="343" />
        <author ui="1" />
    </authors>
    <files>
        <file ui="2" />
    </files>
</book>
\end{verbatim}

Пример добавления к существующей книге автора (если у этой книги уже существовал автор, то он также останется автором этой книги) 
\begin{verbatim}
<book id="223">
    <authors>
        <author ui="1" />
    </authors>
</book>
\end{verbatim}

Для сущности book возможно добавление атрибута reset со значением author или file. 

Его наличие с атрибутом author означает, что только ниже перечисленные авторы написали данную книгу. Если у книги существовали до этого авторы, они будут удалены из списка авторов книги. 
\begin{verbatim}
<book id="276" reset="author">
    <authors>
        <author id="343" />
        <author ui="1" />
    </authors>
</book>
\end{verbatim}

Если атрибут reset имеет значение file, то поведение аналогично вышеописанному только для файлов книги. 

\subsubsection{Обработка ошибок}

Если при обработке запроса произошла ошибка, то ни одно из изменений не будет применено. 
В ответе указывается информация об ошибке.
