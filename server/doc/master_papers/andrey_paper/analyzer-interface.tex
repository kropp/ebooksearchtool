\newpage

Интерфейс к аналайзеру


Важной частью функционирования системы в целом является процесс добваление информации на сервер со стороны аналайзера. Этот процесс состоит из двух этапов. 

Первый этап это распознование аналайзеровм названия книги, её авторов и прочей информации о книге. На этом этапе аналайзер обращается к серверу с запросами вида: есть ли в базе автор с именем похожим на заданное, какие книги заданного автора есть в базе.

Второй этап, это добавление информации на сервер.


Алгоритм взаимодействия.

В разных файлах одной и той же книги могут быть по разному записаны имена авторов. Для поддержания базы в корреектном состоянии необходимо распозновать такие неточности.

К сущностям автор, книга, файл книги было добавлено новое понятие {\em индекс доверия}. Это понятие характеризует проверенность информации о сущности.

Также введено понятие {\em индекс релевантности}. Этот индекс характеризует похожесть двух строк.



Представление алгоритма в виде таблицы

h -- значение выше порога; l -- значение ниже порога; any -- любое значение; add -- добавляем к существующей сущности; new -- создаём новую сущность;

\begin{center}
  \begin{tabular}{ | c | c | c | c | c || c | c |}
  \hline
   & \multicolumn{2}{c|}{Author} & \multicolumn{2}{c||}{Book} & \multicolumn{2}{c|}{Result} \\
    \hline
      \# & Relev & Trust & Relev & Trust & Author & Book \\ \hline
      1 & h   & h   & h   & h   & add & add \\ \hline
      2 & h   & h   & any & l   & add & new \\ \hline
      3 & any & l   & any & any & new & new \\
    \hline
  \end{tabular}
\end{center}



Фаза добавления книги


Для аналайзера реализован интерфейс позволяющий добавлять и модифицировать новые сущности

Запрос состоит из двух секций define и update. 

\lstset{language=XML}
\begin{lstlisting}
<?xml version="1.0" encoding="UTF-8"?>
<request>
    <define>
        ...
    </define>

    <update>
        ...
    </update>
</request>
\end{lstlisting}


Секция define

Здесь необходимо описать создаваемые сущности, но не уже существующие в базе сущности. У каждой сущности должен быть атрибут ui (целое положительное число). Это уникальный идентификатор сущности для этого запроса. В следующей секции по этим ui можно обращаться к сущностям. 

При определении каждой сущности необходимо указать обязательную информацию: 
для author -- full\_name 
для file -- link, size, type
для book -- title 


Можно добавить необязательную информацию: credit, lang, ... 

Пример определения автора 
\lstset{language=XML}
\begin{lstlisting}
<author ui="1">
    <full_name> Leo Tolstoy </full_name>
</author>
\end{lstlisting}

Пример определения файла книги 
\lstset{language=XML}
\begin{lstlisting}
<file ui="2">
    <link>http://example.com</link>
    <type>pdf</type>
    <size>4563214</size>
</file>
\end{lstlisting}

Пример определения книги 
\lstset{language=XML}
\begin{lstlisting}
<book ui="3">
    <title> Red hat </title>
</book>
\end{lstlisting}
Секция update

В этой секции возможна модификация данных. 

К каждой из трёх сущностей возможен доступ по id, если эта сущность уже существует в базе, либо по ui, если она создается в этом запросе. 

Вся указаная информация о сущности будет либо перезаписана, либо добавлена. 

Пример изменения имени автора 
\lstset{language=XML}
\begin{lstlisting}
<author id="45">
    <full_name> Alexander Pushkin </full_name>
</author>
\end{lstlisting}

Пример создания новой книги 
\lstset{language=XML}
\begin{lstlisting}
<book ui="2">
    <authors>
        <author id="343" />
        <author ui="1" />
    </authors>
    <files>
        <file ui="2" />
    </files>
</book>
\end{lstlisting}

Пример добавления к существующей книге автора (если у этой книги уже существовал автор, то он также останется автором этой книги) 
\lstset{language=XML}
\begin{lstlisting}
<book id="223">
    <authors>
        <author ui="1" />
    </authors>
</book>
\end{lstlisting}

Для сущности book возможно добавление атрибута reset со значением author или file. 

Его наличие с атрибутом author, означает, что только ниже перечисленные авторы написали данную книгу. Если у книги существовали до этого авторы, они будут удалены из списка авторов книги. 
\lstset{language=XML}
\begin{lstlisting}
<book id="276" reset="author">
    <authors>
        <author id="343" />
        <author ui="1" />
    </authors>
</book>
\end{lstlisting}

Если атрибут reset имеет значение file, то поведение аналогично вышеописанному, только для файлов книги. 

Обработка ошибок

Если при обработке запроса произошла ошибка, то ни одно из изменений не будет применено. 

В ответе будет указана информация об ошибке.
