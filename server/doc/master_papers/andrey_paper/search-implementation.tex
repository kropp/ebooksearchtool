\section{Реализация поискового механизма}


\subsection{Выбор инструмента}

Теоретически функциональность поиска можно было реализовать, используя
встроенные средсва СУБД, например, like для MySQL и другое.
Но на практике это неразумно.
Так как встроенные средсва достаточно бедны и их функциаональность сильно ограничена.
Например, они не предоставляют возможность учёта морфологии языка.
Следовательно, пришлось бы самостоятельно реализовывать подобные возможности, что достаточно трудоёмко.
И, не зависимо от пути реализации, скорость работы такого механизма, скорее всего была бы невысока.


Естественно, подобная задача важна и широко распространена.
Пoэтому существуют достаточно много готовых решений, называемых поисковыми системами.
Вот неполный список одних из самых популярных решений:
\begin{enumerate}
    \item The Apache Lucene;
    \item Xapian;
    \item Sphinx;
    \item Яндекс.Сервер.
\end{enumerate}

Все они предоставляют схожую функциональность.

Для реализации интерфейса поискового механизма в этом проекте был выбран Sphinx~\cite{sphinx}. 
Это бесплатная поисковая система с открытым исходным кодом, 
предназначенная для быстрого поиска текста. 
Автор проекта --- россиянин Андрей Аксёнов.

\subsection{Реализация поиска с помощью Sphinx}

Процесс осуществления поиска состоит из двух этапов.

Первый --- это индексация данных, то есть преобразование входных данных и добавление их в некоторую базу.
Затем эта база используется для полнотекстового поиска информации.
Sphinx предоставляет большое число возможных настроек процесса индексации.
Второй этап --- это непосредственно поиск. Sphinx также предлагает богатые возможности тонкой настройки вида поиска.

Sphinx вводит три понятия для пользовательских настроек поведения поисковой системы: источник данных, индекс, поиск.
А также два для административных настроек: индексер, поисковый сервис.
Пользовательские настройки позволяют реализовать требуемую функциональность поиска.

\subsubsection{Релевантный поиск}

Sphinx допускает несколько вариантов сортировки результатов поиска в настройках поиска.
Один из них SPH\_SORT\_RELEVANCE вариант, обеспечивающий сортировку результатов по релевантности.

\subsubsection{Фильтрация результатов}

У каждой книги существует понятие язык, на ктором она написана.
Книга может быть написана только на одном языке (по крайне мере, всегда можно выделить основной язык).
Таким образом отношение между сущностью книга и язык <<один к одному>>.
То есть язык является атрибутом книги. В Sphinx присутствует возможность фильтрации результатов поиска по атрибутам.
Эта необходимо сделать в настройках источника данных. Во-первых, надо указать, откуда брать атрибут (language\_id):
\begin{verbatim}
sql_query = SELECT id, title, language_id FROM book_book
\end{verbatim}
Во-вторых, указать, что по этому атрибуту необходимо будет фильтровать рузультаты:
\begin{verbatim}
sql_attr_uint = language_id
\end{verbatim}

Для тэгов ситуация схожа, но отношение между сущностью книга (или автор) и тэг <<многие ко многим>>.
Поэтому настройки имеют несколько другой вид.

Указание источника атрибута, здесь же указание требования фильтрации по этим атрибутам:
\begin{verbatim}
sql_attr_multi = uint tag_id from query; \
	select b.id, bt.tag_id \
	from book_book as b \
	join book_book_tag as bt on b.id=bt.book_id;
\end{verbatim}


\subsubsection{Поиск с учётом морфологии}

Поиск с учётом морфологии в поисковых системах зачастую реализуется с помощью стемминга. Стемминг \cite{stemming} --- это процесс нахождения основы слова для заданного исходного слова.
Основная идея заключается в неразличении слов, находящихся в раздичных словоформах.
Очевидно, что такое преобразование необходимо как на стадии индексации данных, так и на стадии поиска; в последнем случае преобразование происходит над поисковым запросом.
В Sphinx существует встроенный стемминг для английского и русского языков.
Но есть возможность подключить любой другой алгоритм стемминга.
Включается данная опция в настройках индекса
\begin{verbatim}
morphology = stem_enru
\end{verbatim}
\subsubsection{Поиск с опечатками}

При поиске книг у пользователя есть возможность указать или название книги, или имя автора, или оба параметра. В каждом из этих запросов пользователь может допустить опечатку или ошибку.
Исправление опечаток в запросе при указании названия книги осуществляется с помощью aspell. 
Aspell \cite{aspell} --- это свободная программа для исправления орфографии.
Для проверки орфографии используется словарь. 
Проверяемое слово сравнивается со словами, находящимися в словаре.
В случае, если проверяемое слово определено как неправильное, aspell может предложить варианты исправления. Так как в основе механизма исправления слов лежит словарь, то, очевидно, для различных языков такие словари будут различны. Для aspell существуют словари для более 84х языков \cite{langforaspell},
среди которых есть и русский.

При использовании системы пользователь может искать книги на различных языках. Пользователь может включить фильтрацию по языкам, но может и не указывать язык.
В реализации функции исправления запроса был применён следующий приём.
Если указан язык на котором написан запрос, то программа использует соответствующий словарь; если такового не имеется в системе, то проверка орфографии не производится. Сообщение об отсутствии требуемого словаря записывается в лог.
Таким образом, проанализировав лог, можно понять, какие словари более всего требуются пользователям. Достаточно их установить в систему, и программа в следующие разы будет по необходимости использовать новые словари.

Если язык запроса не указан, то программа пытается автоматически распознать язык.
Пока это реализовано только для русского языка, в остальных случаях используется словарь анлийского языка.

Метод проверки запросов по словарю хорошо работает для обычной лексики. Но с именами нарицательными ситуация несколько хуже.

Если в русском языке чаще можно правильно записать фамилию или имя на слух, то в английском языке это почти всегда непросто.
Поэтому для осуществления поиска по авторам с опечатками в запросе применияется совершенно другая идея.
Для решения задачи поиска имён по звучанию используется алгоритм сравнения двух строк по их звучанию.
Основная идея таких алгоритмов заключается в сопоставлении слову некоторого ключа, характеризующего его звучание, а не написание.
Подобных алгоритмов существует несколько видов: Soundex \cite{soundex}, Metaphone \cite{metaphone}, Double Metaphone.

Sphinx имеет встроенную поддержку как и Soundex, так и Metaphone. Подобная опция устанавливается в настройках индекса
\begin{verbatim}
morphology = soundex
\end{verbatim}

Но при таком решении возникает небольшая проблема: в результатах поиска автор, имя которого полностью совпадает с поисковым запросом, может оказаться не на первом месте. А на первом месте может оказаться автор, имя которого звучит также как и запрос.

Дабы решить эту проблему была применена следующая идея. 
Сначала авторы ищутся в индексе, с отключенной морфологией, после --- в индексе с включённой морфологией.
Затем эти результаты объединяются таким образом, что вначале идут авторы из первого результата, а после --- из второго.

\subsubsection{Проблема диакритических знаков}

Как и в названиях книг, так и в именах их авторов могут встречаться символы с различными надстрочными, подстрочными знаками, называемыми {\em диакритическими знаками}. Естественно, пользователь может указать подобные символы в поисковом запросе.
Есть мысль, что если в запросе есть диакритические знаки, то искать хочется с учётом оных. С одной стороны, это так: пример тому слово <<marché>> (фр. <<рынок>>) и <<marche>> (фр. <<ходит>>). 

Но тут возникает проблема. Пользователь может указать в слове (или запросе) только часть необходимых диакритических знаков, тогда, скорее всего, он получит результат хуже, чем при запросе без диакритических знаков. 

Для такого поведения пользователя есть несколько причин:
\begin{enumerate}
    \item Просто лень, невнимательность;
    \item Отсутствие требуемого символа на клавиатуре (француз ищет книгу на испанском);
    \item Пользователь не знает правильное написание фамили автора, но правильно указал диакритический знак в названии книги.
\end{enumerate}

Поэтому было принято решение: не различать буквы с диакритическими знаками и без.

В программе в работе со строками используется Юникод.
В Юникоде символы, имеющие дополнительные над- или подстрочные элементы, 
могут быть представлены в виде построенной по определённым правилам последовательности кодов (составной вариант, composite character) \cite{compositechar} 
или в виде единого символа (монолитный вариант, precomposed character).

Для игнорирования диакритических знаков в составном варианте достаточно указать в настройках индекса Sphinx <<игнорировать модифицирующие символы>>
\begin{verbatim}
ignore_chars = U+0300, U+0301, U+0302, U+0303, ...
\end{verbatim}
Для игнорирования диакритики в едином символе используется возможность Sphinx в настройках индекса определять таблицу символов. Она позволяет задавать правила для отображения одних сиволов в другие.
Эти правила будут использоваться как для данных при индексации, так и для поисковых запросов при поиске.
\begin{verbatim}
charset_table = U+00C0->a, U+00C1->a, ...
\end{verbatim}

\subsubsection{Простой поиск}

Для обеспечения быстрой и удобной работы с базой важную роль играет {\em простой поиск}.
Он позволяет пользователю быстро и удобно получить необходимую информацию.

Пользователю предлагается только одно поле для ввода, в которое он может ввести смешанный запрос.
В таком запросе он может указать как и название книги, так и автора, так и то и другое.
В каждом случае необходимо вернуть корректные результаты.
Поисковый запрос пользователя может быть одним из трёх вариантов:
\begin{enumerate}
	\item название книги, автор;
	\item название книги;
	\item автор.
\end{enumerate}
Так как от пользователя приходит только смешанный поисковый запрос, то узнать какой это из перечисленных вариантов невозможно.
Данная задача была решена следующим образом.

Если поиск среди имён авторов дал плохие результаты, то, видимо, в поисковом запросе нет имени автора. Тогда пользователю возвращаются результаты поиска по названиям книг.

Если поиск среди названий книг не дал результатов, а поиск среди авторов оказался большим, то скорее всего в поисковом запросе указано только имя автора.
В этом случае в качестве результатов поиска возвращается список книг автора, оказавшимся первым в результатах поиска по именам авторов.

В случае, если результаты поиска и по названиям книг, и по авторам дали хорошие результаты, производится фильтрация книг по именам авторов.
Тем самым обеспечивается требуемое поведение в случае, когда в запросе указаны и название книги и её авторы.






