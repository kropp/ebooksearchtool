
\section{Выбор инструмента}


Теоретически функциональность поиска можно было реализовать используя
встроенные средсва СУБД, на пример, like для MySQL и другое.
Но на практике это неразумно, так как реализация подобной задачи 
достаточно трудоёмка.
И не зависимо от пути реализации, скорость работы такого механизма была бы неприемлема низка.
Естественно подобная задача важна и широко распространена.
Пoэтому существуют несколько готовых решений, называемых поисковыми системами.
Вот неполный список одних из самых популярных решений:
\begin{enumerate}
    \item The Apache Lucene 
    \item Xapian
    \item Sphinx
    \item Яндекс.Сервер
\end{enumerate}

Все они предостовляют схожую функциональность.

Для реализации интерфеса поискового механизма в этом проекте был выбран Sphinx. 
Это бесплатная поисковая система с открытым исходным кодом, 
предназначенная для быстрого поиска текста. 
Автор проекта россиянин Андрей Аксёнов.

\section{Реализация поиска с помощью Sphinx}

Процесс реализации поиска состоит из двух этапов.

Первый, это индексация данных, то есть преобразование входных данных и добаления их в некотрую базу.
Затем эта база используется для полнотекстового поиска информации.
Sphinx предоставляет большое число возможных настроек процесса индексации.
Второй этап, это непосредственно поиск. Sphinx также предлагает богатые возможности тонкой настройки вида поиска.

Настройки процессов индексации и поиска позволяет реализовать требуемую функциональность поиска.

\subsection{Релевантный поиск}

Sphinx допускает несколько вариантов сортировки результатов поиска.
Один из них SPH\_SORT\_RELEVANCE вариант, обеспечивающий сортировку результатов по релавантности.

\subsection{Поиск с учётом морфологии}

Поиск с учётом морфологии в поисковых системах зачастую реализуется с помощю стемминга. Стемминг --- это процесс нахождения основы слова для заданного исходного слова.
Основная идея заключается в неразличении слов, находящихся в раздичных словоформах.
Очевидно, что такое преобразование необходимо как на стадии индексации данных, так и на стадии поиска, в последнем случае пеобразование происходит над поисковым запросом.
В Sphinx существует встроенный стемминг для английского и русского языков.
Но и есть возможность подключить любой другой алгоритм стемминга.
Включается данная опция в настройках индексации 

morphology = stem\_enru

\subsection{Поиск с опечатками}

При поиске книг у пользователя есть возможность указать или название книги, или имя автора, или оба параметра. В каждом из этих запросов пользователь может допустить опечатку или ошибку.
Исправление опечаток в запросе при указании названия книги осуществляется с помощью aspell. 
Aspell это свобободная программа для исправления орфографии.
Для проверки орфографии используется словарь. 
Проверяемое слово сравнивается со словами находящимися в словаре.
В случае, если проверяемо слово определено как неправельное, aspell может предложить варинты исправления. Так как в основе механизма исправления слов лежит словарь, то, очевидно, для различных языков такие словари будут различны. Для aspell существуют словари для более чем 84х языков,
среди которых есть и русский.

При использовании системы пользователь может искать книги на различных языках. Пользователь может включить фильтрацию по языкам, но может и не указывать язык.
В реализации функции исправления запроса был применён следующий приём.
Если указан язык на котором написан запрос, то программа использует соответствующий словарь, если такового не имеется в системе, то проверка орфографии не производится. Сообщение об отсутствии требуемого словаря записывается в лог.
Таким образом, проанализировав лог, можно понять какие словари более всего требуются пользователям. Достаточно их установить в систему, и программа в следующие разы будет по необходимости использовать новые словари.

Если язык запроса не указан, то программа пытается автоматически распознать язык.
Пока это реализовано только для русского языка, в остальных случаях используется словарь анлийского языка.

Метод проверки запросов по словарю хорошо работает для обычной лексики. Но с именами нарицательными ситуация несколько хуже.

Если в русском языке чаще можно правильно записать фамилию или имя на слух, то в английском языке это почти всегда непросто.
Поэтому для осуществления поиска по авторам с опечатками в запросе применияется совершенно другая идея.
Для решения задачи поиска имён по звучанию используется алгоритм сравнения двух строк по их звучанию.
Основная идея таких алгоритмов заключается в сопоставлении слову некоторого ключа, характеризующего его звучание, а не написание.
Подобных алгоритмов существует несколько видов: Soundex, Metaphone, Double Metaphone.

Sphinx имеет встроенную поддержку как и Soundex, так и Metaphone. Подобная опция устанавливается в настройках индексатора

morphology = soundex


Но при таком решении возникает небольшая проблема, в результатах поиска автор, имя которого полностью совпадает с поисковым запросом, может оказаться не на первом месте. А на первом месте может оказаться автор, имя которого звучит также как и запрос.

Дабы решить эту проблему было применена следующая идея. 
Сначала авторы исщутся в индексе, с отключенной морфологий, после --- в индексе с включённой морфологий.
Затем эти результаты объединяются таким образом, что вначале идут авторы из первого результата, а после из второго.

\subsection{Проблема диакритических знаков}

Как и в названиях книг, так и в именах их авторов могут встречаться символы с различными надстрочными, подстрочными знаками, называимыми {\em диакритическими знаками}. Естественно, пользователь может указать подобные символы в поисковом запросе.
Есть мысль, что если в запросе есть диакритические знаки, то искать хочется с учётом оных. С одной стороны, это так, пример тому слово "marché" (фр."рынок") и "marche" (фр."ходит"). 

Но тут возникает проблема. Пользователь может указать в слове (или запросе) только часть необходимых диакритических знаков, тогда, скорее всего, он получит результат хуже, чем при запросе без диакритических знаков. 

Для такого поведения пользователя есть несколько причин:
\begin{enumerate}
    \item Просто лень, невнимательность 
    \item Отсутствие требуемого символа на клавиатуре (француз ищет книгу на испанском) 
    \item Пользователь не знает правильное написание фамили автора, но правильно указал диакритический знак в названии книги 
\end{enumerate}

Поэтому, было принято решение --- не различать буквы с диакритическими знаками и без.

В программе в работе со строками используется Юникод.
В Юникоде символы, имеющие дополнительные над- или подстрочные элементы, 
могут быть представлены в виде построенной по определённым правилам последовательности кодов (составной вариант, composite character) 
или в виде единого символа (монолитный вариант, precomposed character).

Для игнорирования диакритических знаков в составном варианте, достаточно указать в настройках индекса Sphinx игнорировать модифицирующие символы

ignore\_chars = U+0300, U+0301, U+0302, U+0303, U+0304, U+0305, ...

Для игнорирования диакритики в едином символе, используется возможность Sphinx определять таблицу символов. Она позволяет задавать правила для отображения одних сиволов в другие.
Эти правила будут использоваться как над данными при индексации, так и над поисковым запросом при поске.

charset\_table = U+00C0->a, U+00C1->a, ...


\subsection{Простой поиск}

Для обеспечения быстрой и удобной работы с базой важную роль играет {\em простой поиск}. 
