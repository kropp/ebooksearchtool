
Выбор инструмента


Теоретически функциональность поиска можно было реализовать используя
встроенные средсва СУБД, на пример, like для MySQL и другое.
Но на практике это невозможно, так как реализация подобной задачи 
достаточно трудоёмка.
Но не зависимо от пути реализации, скорость работы такого механизма была бы неприемлема низка.

Естественно подобная задача важна и широко распространена.
Пoэтому существуют несколько готовых решений, называемых поисковыми системами.
Вот неполный список одних из самых популярных решений:
\begin{enumerate}
    \item The Apache Lucene 
    \item Xapian
    \item Sphinx
    \item Яндекс.Сервер
\end{enumerate}

Все они предостовляют схожую функциональность.

Для реализации интерфеса поискового механизма в этом проекте был выбран Sphinx. 
Это бесплатная поисковая систему с открытым исходным кодом, 
предназначенную для быстрого поиска текста. 
Автор проекта россиянин Андрей Аксёнов.


Реализация поиска с помощью Sphinx

Процесс реализации поиска состоит из двух этапов.

Первый, это индексация данных, то есть преобразование входных данных и добаления их в некотрую базу.
Затем эта база используется для полнотекстового поиска информации.
Sphinx позволяет большое число возможных настроек процесса индексации.

Второй этап это поиск, Sphinx предлашает богатые возможности тонкой настройки вида поиска.

Настройки процессов индексации и поиска позволяет реализовать требуемую функциональность поиска.


Релевантный поиск.

Sphinx допускает несколько вариантов сортировки результатов поиска.
Один из них SPH\_SORT\_RELEVANCE вариант, обеспечивающий требуемое поведение.

Поиск с учётом морфологии.

Поиск с учётом морфологии в поисковых системах зачастую реализуется с помощю стемминга. Стемминг — это процесс нахождения основы слова для заданного исходного слова.
Основная идея заключается в неразличении слов, находящихся в раздичных словоформах.
Очевидно, что такое преобразование необходимо как на стадии индексации данных, так и на стадии поиска, в данном случае пеобразование происодит над поисковым запросом.
В Sphinx существует встроенный стемминг для английского и русского языков.
Но и есть возможность подключить любой другой алгоритм стемминга.
Включается данная опция в настройках индексации 

morphology = stem\_enru


Поиск с опечатками

При поиске книг у пользователя есть возможность указать или название книги, или имя автора, или оба параметра. В каждом из этих запросов пользователь может допустить опечатку или ошибку.
Исправление опечаток в запросе при указании названия книги осуществляется с помощью aspell. 
Aspell это свобободная программа для исправления орфографии.
Для проверки орфографии используется словарь. 
Проверяемое слово сравнивается со словами находящимися в словаре.
В случае, если проверяемо слово определено как не правельное, aspell может предложить варинты исправления. Так как в основе механизма испарвления слов лежит словарь, то очевидно для различных языков такие словари будут различны. Для aspell существуют словари для более чем 84х языков,
среди которых есть и русский.

При использовании системы пользователь может искать книги на различных языках. Пользователь может включить фильтрацию по языкам, но может и не указывать язык.
В реализации функции исправления запроса был применён следующий приём.
Если указан язык на котором написан запрос, то программа использует соответствующий словарь, если такового не имеется в системе, то проверка орфографии не производится. Сообщение об отсутствии требуемого словаря записывается в лог.
Таким образом, проанализировав лог, можно понять какие словари более всего требуется пользователям. Достаточно их установить в систему, и программа в следующие разы будет по необходимости использовать новые словари.

Если язык запроса не указан, то программа пытается автоматически распознать язык.
Пока это реализовано только для русского языка, в остальных случаях используется словарь анлийского языка.

Метод проверки запросов по словарю хорошо работает для обычной лексики. Но с именами нарицательными ситуация несколько хуже.

Если в русском языке чаще можно правильно записать фамилию или имя на слух, то в английском языке это почти всегда непросто.
Поэтому для осоществления поиска по авторам с опечатками в запросе применияется совершенно другая идея.
Для решения задачи поиска имён по звучанию используется алгоритм сравнения двух строк по их звучанию.
Основная идея таких алгоритмов заключается в сопоставлении слову некоторого ключа, характеризующего его произношение, а не написание.
Подобных алгоритмов существует несколько видов: Soundex, Metaphone, Double Metaphone.

Sphinx емеет встроенную поддержку как и Soundex, так и Metaphone. Подобная опция устанавливается в настройках индексатора

morphology = soundex










