
\section{Постановка задачи}

\subsection{Хранение данных}

Необходимо хранить информацию о книгах

\subsection{Поиск по данным}

Если существует некоторая большая база с информацией, то очевидно, 
что для быстрой и удобной работы с ней необходим мощный, быстрый и удобный поиск.

В нашей модели для пользователя есть несколько сущностей: название книги, 
список её авторов, язык, на котором написана книга, тэги, характеризующие её стиль, жанр, содержание.

Ниже сформулированы требования для функции поиска с точки зрения пользователя:
\begin{enumerate}
  \item  Релевантный поиск как по отдельным сущностям, так и по различным их комбинациям;
  \item  Фильтрация результатов поиска по некоторым сущностям (язык книги, тэг);
  \item  Поиск с учётом морфологии языка;
  \item  Поиск с учётом опечаток в запросе или их исправление;
  \item  Простой поиск (простой в использовании).
\end{enumerate}

При разработке поискового механизма необходимо: сохранить слабую связанность отдельных компонентов программы, обеспечить возможность замены реализации поиска с минимальными изменениями в остальном коде, учесть возможность расширения и изменения требований к функции поска.


\subsection{Интерфейс модификации данных}

Должна присутствовать возможность добавления новых данных, 
а также возможность модифицировать существующие.
