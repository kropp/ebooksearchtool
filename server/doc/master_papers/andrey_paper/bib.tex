\cleardoublepage
\section{Библиографический список}
 
\renewcommand*{\refname}{}
\begin{thebibliography}{99}

% For head
\bibitem{googleBook} Google Books \url{http://books.google.com/}

\bibitem{ebdb} eBdb --- electronics books data base \url{http://ebdb.ru/}

\bibitem{kindle} Amazon Kindle \url{http://amazon.com/kindle/}

\bibitem{sonyreader} Sony Reader \url{http://www.learningcenter.sony.us/assets/itpd/reader/}

\bibitem{opds} The Open Publication Distribution System (OPDS) \url{http://code.google.com/p/openpub/}

\bibitem{bookserver} BookServer \url{http://bookserver.archive.org/}

\bibitem{archive} Internet Archive \url{http://archive.org/}

\bibitem{feedbooks} feedbooks \url{http://feedbooks.com/}

\bibitem{django} Django веб-фреймворк \url{http://www.djangoproject.com/}

\bibitem{djangomvc} Описание архитектуры Django сайтов на DjangoBook \url{http://www.djangobook.com/en/2.0/chapter01/}

% For seatch implementation
\bibitem{sphinx} Поисковая система Sphinx \url{http://sphinx.com/}

\bibitem{aspell} Aspell \url{http://aspell.net/}

\bibitem{langforaspell} Aspell поддерживаемые языки \url{http://aspell.net/man-html/Supported.html}

\bibitem{stemming} Страница про стемминг на английской Wikipedia \url{http://en.wikipedia.org/wiki/Stemming/}

\bibitem{soundex} Страница про soundex на английской Wikipedia \url{http://en.wikipedia.org/wiki/soundex/}

\bibitem{metaphone} Страница про metaphone на английской Wikipedia \url{http://en.wikipedia.org/wiki/Metaphone/}

\bibitem{compositechar} Список модифицирующих символов из The Unicode Standard, Version 5.2.  \url{http://www.unicode.org/charts/PDF/U0300.pdf/}

\bibitem{distance} Страница про расстояние Левенштейна на русской Wikipedia \url{http://ru.wikipedia.org/wiki/Расстояние_Левенштейна}


\end{thebibliography}